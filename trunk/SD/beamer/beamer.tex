\documentclass{beamer}

%
% Common preamble for all three parts.
%

\usepackage[spanish]{babel}
\usepackage{amsmath}
\usepackage{color}
\ProvidesPackage{minted}
\usepackage{minted}
%\setminted{encoding=utf8}
\usepackage{hyperref}
\usepackage{multicol}
\usepackage{tabularx}
\usepackage{tikz}

\usepackage[utf8x]{inputenc}
%\usepackage{ucs}
%\usepackage[T1]{fontenc}
%\newcommand{\minted@encoding}{\minted@get@opt{encoding}{UTF8}}

% o Warsaw, Bergen, Madrid, ...
%\usetheme{Berkeley}
\usetheme{Warsaw}

% o albatross, beaver, crane, ...
% \usecolortheme{whale}
\definecolor{IUAblue}{RGB}{53,62,91}
\definecolor{IUAamarillo}{RGB}{254,187,80}


\setbeamercolor*{palette primary}{use=structure,fg=white,bg=IUAblue}
\setbeamercolor*{palette quaternary}{fg=white,bg=IUAamarillo}%black!30!IUAamarillo}

%\setbeamercolor{structure}{bg=IUAamarillo, fg=IUAamarillo!90!IUAblue}

% no nav buttons
%\usenavigationsymbolstemplate{}

\newcommand{\bftt}[1]{\textbf{\texttt{#1}}}
\newcommand{\comment}[1]{{\color[HTML]{008080}\textit{\textbf{\texttt{#1}}}}}
\newcommand{\cmd}[1]{{\color[HTML]{008000}\bftt{#1}}}
\newcommand{\bs}{\char`\\}

%\logo{\includegraphics[width=.125\textwidth]{images/logo_f_azul}}



\title[Proxy]{Patrón de diseño Proxy}
\author{Luis A. Guanuco}
\institute[ESE -- IUA]{Especialización en Sistemabas Embebidos \\ Instituto Universitario Aeronáutico}
\date{11 de Marzo del 2016}
\titlegraphic{%
\includegraphics[width=.2\textwidth]{images/logo_f_blanco}
}

\subtitle{Algoritmo y Patrones de Software}

\begin{document}

%%%%%%%%%%%%%%%%%%%%%%%%%%%%%%%%%%%%%%%%%%%%%

\begin{frame}
  \titlepage
\end{frame}

%%%%%%%%%%%%%%%%%%%%%%%%%%%%%%%%%%%%%%%%%%%%%
\section[Descripción]{Descripción}
%%%%%%%%%%%%%%%%%%%%%%%%%%%%%%%%%%%%%%%%%%%%%

\begin{frame}{\insertsection{}}
  \frametitle{Propósito}
    Proporcionar un representante o sustituto de otro objeto
    para controlar el acceso a éste
    \vfill
    \centering
    \includegraphics[width=.6\textwidth]{images/proxy-example}
\end{frame}

\begin{frame}{\insertsection{}}
  \frametitle{Motivación}
  Retrasar el coste de crear e inicializar un objeto hasta
  que es realmente necesario. Por ejemplo, no abrir las
  imágenes de un documento hasta que no son visibles
  \vfill
  \centering
  \includegraphics[width=.6\textwidth]{images/problema}
\end{frame}

\begin{frame}{\insertsection{}}
  \frametitle{Motivación}
  \includegraphics[width=\textwidth]{images/solucion}  
\end{frame}

%%%%%%%%%%%%%%%%%%%%%%%%%%%%%%%%%%%%%%%%%%%%%
\subsection[Aplicatibilidad]{Aplicatibilidad}
%%%%%%%%%%%%%%%%%%%%%%%%%%%%%%%%%%%%%%%%%%%%%

\begin{frame}{\insertsubsection{}}
  \frametitle{\insertsubsection{}}
  El patrón Proxy se usa cuando se necesita una referencia
  a un objeto más flexible o sofisticada que un puntero.

  Algunas situaciones comunes de aplicación son:
  \begin{description}
  \item[Proxy remoto:] representa un objeto en otro espacio de
    direcciones.
  \item[Proxy virtual:] crea objetos costosos por encargo (como la clase
    \texttt{ImageProxy} en el ejemplo de motivación).
    \item[Proxy de protección:] controla el acceso a un objeto.
    \item[Referencia inteligente:] sustituto de un puntero que lleva a
      cabo operaciones adicionales cuando se accede a un objeto
      (ej. contar el número de referencias, cargar un objeto
      persistente en memoria, bloquear el objeto para impedir acceso
      concurrente, ...).
  \end{description}
\end{frame}

%%%%%%%%%%%%%%%%%%%%%%%%%%%%%%%%%%%%%%%%%%%%%
\subsection[Estructura]{Estructura}
%%%%%%%%%%%%%%%%%%%%%%%%%%%%%%%%%%%%%%%%%%%%%

\begin{frame}{\insertsubsection{}}
  \frametitle{\insertsubsection{}}
  \includegraphics[width=\textwidth]{images/estructura}
\end{frame}

%%%%%%%%%%%%%%%%%%%%%%%%%%%%%%%%%%%%%%%%%%%%%
\section[Presentación]{Presentación}
%%%%%%%%%%%%%%%%%%%%%%%%%%%%%%%%%%%%%%%%%%%%%

\begin{frame}{\insertsection{}}
  \frametitle{Proxy (\texttt{ImageProxy})}
  \begin{itemize}
  \item Mantiene una referencia al objeto real
  \item Proporciona una interfaz idéntica a la del objeto real
  \item Controla el acceso al objeto real, y puede ser responsable de
    crearlo y borrarlo
  \item Otras responsabilidades dependen del tipo de proxy:
    \begin{itemize}
    \item Proxies remotos: codifican las peticiones, y las envían al
      objeto real
    \item Proxies virtuales: pueden guardar información del objeto
      real (caché)
    \item Proxies de protección: comprueban que el cliente tiene los
      permisos necesarios para realizar la petición
    \end{itemize}
  \end{itemize}
\end{frame}

\begin{frame}{\insertsection{}}
  \frametitle{Subject (\texttt{Graphic})}
  Define una interfaz común para el proxy y el objeto real, de tal
  modo que se puedan usar de manera indistinta
\end{frame}

\begin{frame}{\insertsection{}}
  \frametitle{RealSubject (\texttt{Image})}
  Clase del objeto real que el proxy representa
\end{frame}

%%%%%%%%%%%%%%%%%%%%%%%%%%%%%%%%%%%%%%%%%%%%%
\section[Ventajas]{Ventajas}
%%%%%%%%%%%%%%%%%%%%%%%%%%%%%%%%%%%%%%%%%%%%%

\begin{frame}{\insertsection{}}
  \frametitle{\insertsection{}}

  Introduce un nivel de indirección con diferentes usos:
  \begin{itemize}
  \item Un proxy remoto puede ocultar el hecho de que un objeto
    reside en otro espacio de direcciones 
  \item Un proxy virtual puede realizar optimizaciones, como la
    creación de objetos bajo demanda 
  \item Los proxies de protección y las referencias inteligentes
    permiten realizar tareas de mantenimiento adicionales al acceder
    a un objeto
  \end{itemize}
\end{frame}  
  
\begin{frame}{\insertsection{}}
  \frametitle{\insertsection{}}
  Optimización copy-on-write
  \begin{itemize}
  \item Copiar un objeto grande puede ser costoso
  \item Si la copia no se modifica, no es necesario incurrir en
    dicho gasto
  \item El sujeto mantiene un número de referencias, y sólo cuando
    se realiza una operación que modifica el objeto, éste se copia
  \end{itemize}    
\end{frame}  

\section{Patrones Relacionados}

\begin{frame}{\insertsection{}}
  \frametitle{Patrón Adaptor}
  El patrón Adaptador proporciona una interfaz diferente al objeto que
  adapta. Ambos patrones redirigen la petición del cliente además de
  incorporar lógica adicional : comprobación de   acceso, creación del
  sujeto real.
\end{frame}  


\begin{frame}{\insertsection{}}
  \frametitle{Patrón Decorador}
  El Proxy se puede diseñar de manera similar al patrón decorador,
  pero el propósito es diferente: el decorador añade responsabilidades
  a un objeto, el proxy sólo controla su acceso
\end{frame}  


\section{Usos conocidos}

\begin{frame}{\insertsection{}}
  \frametitle{\insertsection{}}
  \begin{description}
  \item[NeXTSETP\footnote{\texttt{https://es.wikipedia.org/wiki/NEXTSTEP}}:]
    Combinado con el patrón Forwarder-Receiver se implementa el
    concepto de \emph{stub}.NeXSTEP implementa para proveer stubs
    locales a objetos remotos.

  \item[OMG-CORBA\footnote{\texttt{https://es.wikipedia.org/wiki/CORBA}}:]
    utiliza el patrón Proxy para la implementación de
    \emph{client-stubs (o IDL-stubs)}  y \emph{IDL-skeletons}. Dónde
    IDL es el lenguaje de definción de interfaz definido por el OMG
    (Object Management Group).

\item[RMI\footnote{\texttt{http://www.informit.com/articles/article.aspx?p=1398608\&seqNum=3}}:] Implementación de Java para invocar métodos de objetos
  remotos,\emph{Java Remote Method Invocation (RMI)}.
  \end{description}
\end{frame}  
\end{document}
