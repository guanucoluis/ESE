\documentclass[11pt,oneside,spanish,a4paper]{article}
\usepackage[utf8]{inputenc}
\usepackage[spanish]{babel}
\usepackage[a4paper]{geometry}
\usepackage{graphicx}
\usepackage{fancyhdr}
\usepackage[hyphenbreaks]{breakurl}
\usepackage[hyphens]{url}
\usepackage{hyperref}
\usepackage{listings}
\usepackage{lstlinebgrd}
\usepackage{xcolor}
\usepackage{multicol}
\usepackage{pdfpages}

%%%%%%%%%%%%%%%%%%%%
% Nuevos comandos  %
%%%%%%%%%%%%%%%%%%%%
\newcommand{\HRule}{\rule{\linewidth}{0.5mm}}

\hypersetup{
    pdfcreator={pdfLaTeX},   % creator of the document
    colorlinks=true,       % false: boxed links; true: coloblue links
    linkcolor=blue,          % color of internal links (change box color with linkbordercolor)
    citecolor=blue,        % color of links to bibliography
    filecolor=blue,      % color of file links
    urlcolor=blue           % color of external links
}

\pagestyle{fancy}
\addtolength{\textheight}{2cm}
%\addtolength{\voffset}{-1cm}
%\addtolength{\textwidth}{1cm}

\definecolor{light-gray}{gray}{0.9}

\lstset{
  basicstyle=\ttfamily\color{red},%\footnotesize,%\scriptsize,
  language=tcl,
  breaklines=true,
  otherkeywords={+++},
  numbers=left,
  numberstyle=\scriptsize\color{black}
}

\lstset{literate=
  {á}{{\'a}}1 {é}{{\'e}}1 {í}{{\'i}}1 {ó}{{\'o}}1 {ú}{{\'u}}1
  {Á}{{\'A}}1 {É}{{\'E}}1 {Í}{{\'I}}1 {Ó}{{\'O}}1 {Ú}{{\'U}}1
  {à}{{\`a}}1 {è}{{\'e}}1 {ì}{{\`i}}1 {ò}{{\`o}}1 {ù}{{\`u}}1
  {À}{{\`A}}1 {È}{{\'E}}1 {Ì}{{\`I}}1 {Ò}{{\`O}}1 {Ù}{{\`U}}1
  {ä}{{\"a}}1 {ë}{{\"e}}1 {ï}{{\"i}}1 {ö}{{\"o}}1 {ü}{{\"u}}1
  {Ä}{{\"A}}1 {Ë}{{\"E}}1 {Ï}{{\"I}}1 {Ö}{{\"O}}1 {Ü}{{\"U}}1
  {â}{{\^a}}1 {ê}{{\^e}}1 {î}{{\^i}}1 {ô}{{\^o}}1 {û}{{\^u}}1
  {Â}{{\^A}}1 {Ê}{{\^E}}1 {Î}{{\^I}}1 {Ô}{{\^O}}1 {Û}{{\^U}}1
  {œ}{{\oe}}1 {Œ}{{\OE}}1 {æ}{{\ae}}1 {Æ}{{\AE}}1 {ß}{{\ss}}1
  {ç}{{\c c}}1 {Ç}{{\c C}}1 {ø}{{\o}}1 {å}{{\r a}}1 {Å}{{\r A}}1
  {€}{{\EUR}}1 {£}{{\pounds}}1 {"}{{``}}1
}

\begin{document}

%%%%%%%%%%%%%%%%%%%%%%%%%
% Carátula del informe  %
%%%%%%%%%%%%%%%%%%%%%%%%%

\begin{titlepage}
\begin{center}

\textsc{\LARGE Instituto Universitario Aeronáutico}\\[0.5cm]
\textsc{\LARGE Especialización en Sistemas Embebidos}\\[2cm]

\includegraphics[width=0.2\textwidth]{img/logo_f_blanco}~\\[2cm]

\textsc{\Large Entornos Inalámbricos}\\[0.5cm]

\HRule \\[0.4cm]
{ \huge \bfseries Trabajos Prácticos \\[0.4cm] }

\HRule \\[1.5cm]

% Author and supervisor
\begin{minipage}{0.4\textwidth}
\begin{flushleft} \large
\emph{Alumno:}\\
Luis Alberto \textsc{Guanuco}\\
Santiago Nicolás \textsc{Nolasco}\\
Sebastian \textsc{Aguero}\\
Franco \textsc{Bocalon}
\end{flushleft}
\end{minipage}
\begin{minipage}{0.4\textwidth}
\begin{flushright} \large
\emph{Docentes:} \\
Jose \textsc{Duclox}
\end{flushright}
\end{minipage}
\vfill
{\large Septiembre 2016}

\end{center}
\end{titlepage}

%\lhead{Luis A. Guanuco}
%\chead{\includegraphics[width=0.02\textwidth]{images/logoUTN}}
%\rhead{Arquitectura Embebidas y Proceso de Tiempo Real}
\section{Introducción}
\label{sec:intro}

Se realizarán ejercitaciones sobre plataformas XBee con el objetivo
de entender los \emph{Entornos Inalámbricos}. Se trabajará con los
modos \emph{AT} y \emph{API} a fin de comprar las ventajas y
desventajas de estos. Estos ejemplos prácticos servirán, además,
alcanzar la resolución del \emph{Trabajo Final}.

\subsection{Módulos XBee S2C}
\label{sec:xbee-mod}

Para el establecimiento de una red basado en los estándares 802.15.4
se plantea como requisito la disponibilidad de módulos que implementen
dicho estándar. En nuestro caso se utilizarán los módulos \emph{XBee
  S2C}\footnote{\burl{http://www.digi.com/support/productdetail?pid=4838}}. Las
principales características de estos dispositivos son:
\begin{itemize}
\item Velocidades de datos
  \begin{itemize}
  \item RF 250 Kbps
  \item Comunicación serial hasta 1 Mbps
  \end{itemize}
\item Alcances
  \begin{itemize}
  \item \textsl{Indoor} 60 metros
  \item \textsl{Outdoor} 1200 metros
  \end{itemize}
\item Potencia de transmisión 3.1 mW (+5 dBm). En modo \textsl{Boost}
  6.3 mW (+8 dBm)
\item Sensibilidad de recepción -100 dBm. EN modo \textsl{Boost} -102
  dBm
\item Banda de frecuencia 2.4 GHz (16 canales)
\item 15 puertos digitales I/O (4 entradas analógicas)
\item Alimentación 2.1V a 3.6V. Consumos:
  \begin{itemize}
  \item En la transmisión 33 mA @ 3.3 VDC. En modo \textsl{Boost}45 mA
  \item En la recepción 28 mA @ 3.3 VDC. En modo \textsl{Boost} 31 mA
  \end{itemize}
\end{itemize}

Los puntos anteriores solo describen las principales características
de los módulos XBee 2SC. Para conocer más en detalle se puede acceder
a la hoja de datos disponible en el sitio web del fabricante
\cite{s2c-ds}.

\subsection{Plataformas adicionales}
\label{sec:plat}

Se utilizará \textsl{hardware} adicional que permita establecer la
comunicación de los módulos XBee con una computadora. Sí bien el
Laboratorio de Sistemas embebidos pone a disposición las plataformas
XBoard\footnote{\burl{http://www.cika.com/soporte/Information/Digi-RF/XBee/}}
se necesita adaptar la comunicación serial del módulo XBee con la la
computadora mediante una puerto USB. Para esto se desarrolló la placa
\emph{XBee-MCP Adapter} (Figura \ref{fig:xb-mcp-adap}). Sobre esta
placa se montarán dos módulos, una es el módulo XBee y el otro es la
placa \emph{MCP2200 Breakout
  Module}\footnote{\burl{http://www.microchip.com/DevelopmentTools/ProductDetails.aspx?PartNO=adm00393}}. Este
último proporciona una comunicación serial USB-UART. 
Además se agregaron dos LEDs conectados a los puertos
\texttt{AD1/DIO1} y \texttt{AD2/DIO2}. Opcionalmente se puede conectar
un potenciómetro al conector \texttt{K1} que se encuentra mapeado al
puerto analógico de la XBee, \texttt{AD0/DIO0}.

\begin{figure}[ht]
  \centering
  \includegraphics{img/xb-mcp-adapter}
  \caption{Placa XBee-MCP Adapter.}
  \label{fig:xb-mcp-adap}
\end{figure}

La Xboard, anteriormente mencionada, es una plataforma desarrollada
por la empresa Cika Electrónica SRL. La XBoard permite interconectar
los módulos XBee con dispositivos externos \cite{xboard}. En la Figura
\ref{fig:xboard} se presenta una vista superior de la XBoard. Se puede
ver que esta placa tiene un conector hembra con las señales de
comunicación serial, niveles de tensión y otras más. Para conectar la
XBoard a una computadora se utiliza la placa MCP2200 Breakout
Module. Al no requerir componentes externos, se utilizan simplemente
cables que realicen la interconexión de las señales de comunicación
serial.

\begin{figure}[ht]
  \centering
  \includegraphics[width=.6\textwidth]{img/xboard}
  \caption{Placa XBoard.}
  \label{fig:xboard}
\end{figure}

Todas las características y configuraciones de las diferentes
plataformas comerciales  se encuentran en la bibliografía referenciada
al final de este documento. 

Se desarrollarán tres tipos de ejercitaciones. La primera sobre los
alcances de RF que proporcionan estos dispositivos. La segunda parte
sobre el manejo de los módulos en modo AT. Y por último se trabajará
en modo API con los módulos sobre una red ZigBee completa. Se
documentarán todos las implementaciones y por último se presentará una
conclusión sobre lo hecho.

\section{Análisis de enlaces}
\label{sec:enlace}

Completar con lo de santiago.

\section{Ejercitación en modo AT}
\label{sec:AT}

El set de comandos AT, también conocidos como set de comandos Hayes,
fue originalmente desarrollado para usar con los modem de Hayes en la
década de 1980. Muchos modems modernos todavía utilizan este
estándar. EL término \emph{comando AT}  viene de usar los caracteres
ASCII para notificar al \textsl{host}  que un le sigue un comando. 

En el caso de los módulos XBee desarrollados por Digi, implementa un
set de comandos propieratrios para interactuar con los módulos de
Digi a través de una comunicación serial. Basado en el set de comandos
AT, Dii usa los caracteres \texttt{AT} antes de cada comando enviado
al modem. Un ejemplo de un comando utilizado por el radio Digi XBee
podría ser \texttt{ATCH}. Este comando es usado para leer o setear el
canal que un radio XBeee es configurado\cite{at-cmds}.

\subsection{Configuración de puertos GPIO}
\label{sec:config-at}

La configuración de los diferentes canales de propósitos generales se
puede realizar utilizando el software XCTU en modo gráfico pero el
objetivo es realizarlo mediante el set de comandos AT. 

\subsubsection{Comando ATIS}
\label{sec:atis}

El comando \texttt{ATIS} fuerza la lectura de todos los canales
habilitados. Analógicos como digitales. En nuestro caso tenemos la
siguiente salida a la petición.

\begin{lstlisting}[emph={+++,ATIS}, emphstyle={\color{blue}}]
+++
OK
ATIS
01
0006
01
0000
0219
\end{lstlisting}

La respuesta que entrega el módulo comienza en la línea \textbf{4}
hasta la \textbf{8}. El primer byte \texttt{01} es la cantidad de
muestra recibidas. En la línea \textbf{5} se tiene configuración de
los canales digitales y la línea \textbf{6} canales analógicos. Para
entender se debe analizar a nivel de bits cada una de las
respuestas. Recuerde que los valores representados están en
hexadecimal. En el primer se tiene $0006_{HEX} =
000000000000110_{BIN}"$ por lo los canales \texttt{DIO1} y
\texttt{DIO2} se encuentran habilitados y configurados como salidas
digitales. Mientras que para los canales analógicos tenemos
$01_{HEX} = 0001_{BIN}$ solo el puerto \texttt{AD0} habilitado. Para
terminar el análisis de la respuesta al comando \texttt{ATIS}, las
líneas \textbf{7} y \textbf{8} son el estado de los canales
digitales y analógicos respectivamente. Como se mencionó
anteriormente, puede visualizarse la configuración de los canales de
I/O en forma gráfica desde el software XCTU. En la Figura
\ref{fig:xctu-setup} se puede ver la misma información que
proporciona el comando \texttt{ATIS}.

\begin{figure}[ht]
  \centering
  \includegraphics[width=.8\textwidth]{img/xctu-atis}
  \caption{Configuración de los GPIO en modo gráfico.}
  \label{fig:xctu-setup}
\end{figure}

Para dar comienzo con la ejercitación pedida por los docentes, se
deshabilitarán todos los canales de forma tal que el módulo quede sin
ningún GPIO en uso. Sí la configuración es tal como la descrita
anteriormente, se deben aplicar los comandos siguientes.

\begin{multicols}{2}
\begin{lstlisting}[emph={+++,ATIS,ATD00,ATD10,ATD20,ATWR,ATAC}, emphstyle={\color{blue}}]
+++OK
ATD00
OK
ATD10
OK
ATD20
OK
ATAC
OK
ATWR
OK
ATIS
ERROR
\end{lstlisting}

Para deshabilitar un canal digital/analógico se debe simplemente pasar
como último argumento el valor \texttt{0}. Esto se aplica a los
canales \texttt{DIO0}, \texttt{DIO1} y \texttt{DIO2}. Los comandos que
le siguen son para aplicar los cambios y liberar los \textsl{buffers}
(\texttt{ATAC}) y  escribir la
memoria no-volatil del módulo (\texttt{ATWR}). Al finalizar los
cambios se envía el comando \texttt{ATIS} y se tiene como respuesta
\texttt{ERROR} esto no se debe a un problema de comunicación sino que
al no existir ningún canal habilitado, no puede ofrecer información
alguna. 
\end{multicols}

\subsubsection{Configurar dos canales analógicos}

\section{Ejercitación en modo API}
\label{sec:API}


En la Sección \ref{sec:intro-des} se listó los requerimientos del
proyecto. Una tarea del RTOS debía ser la encargada de realizar la
lectura de una señal analógica cada 100 Hz. Las características para
esta tarea se definen en el lenguaje OIL.
\begin{lstlisting}[numbers=left,title=Declaración del objeto
\texttt{Analogic} en el lenguaje OIL, frame=single]
TASK Analogic {
    PRIORITY = 1;
    ACTIVATION = 1;
    STACK = 512;
    TYPE = EXTENDED;
    SCHEDULE = FULL;
    RESOURCE = POSIXR;
    EVENT = POSIXE;
    AUTOSTART = TRUE {
       APPMODE = AppMode1;
       ALARMTIME = 1;
       CYCLETIME = 1;
    }
}

RESOURCE = POSIXR;
EVENT = POSIXE;

ALARM AnalogicAlarm {
    COUNTER = SoftwareCounter;
    ACTION = ACTIVATETASK {
        TASK = Analogic;
    }
}
\end{lstlisting}
La periodicidad del evento se
configura en el archivo \texttt{main.c} del proyecto. En la línea
\textbf{20} se define la base de tiempo será un contador generado por
el OS es decir \texttt{SoftwareCounter}.

\begin{lstlisting}[numbers=left,title=Código de la tarea
\texttt{Analogic} en el \texttt{main.c}, frame=single]
TASK(Analogic)
{
   uint16_t hr_ciaaDac;
   uint8_t outputs;

   /* Read ADC. */
   ciaaPOSIX_read(fd_adc, &hr_ciaaDac, sizeof(hr_ciaaDac));

   /* Signal gain. */
   hr_ciaaDac >>= 0;

   /* Write DAC */
   ciaaPOSIX_write(fd_dac, &hr_ciaaDac, sizeof(hr_ciaaDac));

   TerminateTask();
}
\end{lstlisting}

\section{Conclusiones}
\label{sec:conc}

La implementación de este RTOS tan específico y novedoso, por lo menos
para nosotros, nos ha permitido ampliar la concepción de sistemas
operativos en tiempo real. A la vez que se ha podido interactuar con
profesionales de la región que están en pleno desarrollo y aportes
para mejorar el código utilizado (freeOSEK). 

\begin{thebibliography}{1}
\bibitem{s2c-ds}
  Digi International,
  \emph{XBee\textsuperscript{\textregistered{}}/XBee-PRO
    ZigBee\textsuperscript{\textregistered{}} RF Module -- User
    Guide}. 2016.
  
\bibitem{xboard}
  Cika Electrónica SRL.,
  \emph{XBoard [-W]}.
  \burl{http://www.cika.com/soporte/Information/Digi-RF/XBee/XBoard_doc.pdf}.

\bibitem{at-cmds}
  Digi International,
  \emph{Knowledge Base -- The AT Command Set}.
  \burl{http://knowledge.digi.com/articles/Knowledge_Base_Article/The-AT-Command-Set}.

\end{thebibliography}

% \appendix{}

% \chapter{Códigos del programa}
% \label{chap:code-prog}

% \lstinputlisting[basicstyle=\scriptsize,title=Código de la función
% \texttt{main.c}]{src/examples/adc_dac/src/adc_dac.c}

% \lstinputlisting[basicstyle=\scriptsize,title=Código del archivo
% \texttt{proyectofinal.oil}]{src/examples/adc_dac/etc/adc_dac.oil}

% \chapter{Esquemáticos del hardware utulizado}
% \label{chap:sch}

% \includepdf[landscape=true,pages=-]{img/Schematic}

\end{document}
