% Also note that the "draftcls" or "draftclsnofoot", not "draft", option
% should be used if it is desired that the figures are to be displayed in
% draft mode.
%
\documentclass[conference]{IEEEtran}
%
\ifCLASSINFOpdf
\usepackage[pdftex]{graphicx}
  % \DeclareGraphicsExtensions{.pdf,.jpeg,.png}
\else
  % \DeclareGraphicsExtensions{.eps}
\fi

% *** MATH PACKAGES ***
%
\usepackage[cmex10]{amsmath}

%\usepackage[caption=false]{caption}
\usepackage[font=footnotesize,caption=false]{subfig}

% --------------- USEPACKAGE agregados por guanucoluis ----------------

\usepackage[utf8]{inputenc}
\usepackage{multirow}
%\usepackage[english]{babel}
\usepackage{amssymb}
%\usepackage[pdftex]{graphicx}
\usepackage[hyphenbreaks]{breakurl}
\usepackage[hyphens]{url}

% ------------------------- Agregados por maxi ------------------------

\renewcommand{\abstractname}{Resumen}
\renewcommand{\figurename}{Fig.}
\renewcommand{\tablename}{Tabla}
\renewcommand{\refname}{Referencias}
\hyphenation{de-sa-rro-llar de-sa-rro-llos de-sa-rro-llo clas-si-fi-can ne-ce-sa-ria-men-te dis-po-si-ti-vos in-te-gra-das es-pa-cio pre-sen-tan di-men-sio-nes di-fe-ren-tes in-dus-tri-al prin-ci-pa-les per-mi-ten com-pu-ta-do-ras pro-por-cio-na dis-po-si-ti-vo im-ple-men-tar par-ti-ci-pa-do di-gi-ta-les rui-do-sa he-rra-mien-tas}

% correct bad hyphenation here
\hyphenation{op-tical net-works semi-conduc-tor}


\begin{document}
%
% paper title
% can use linebreaks \\ within to get better formatting as desired
\title{Revisión -- Bubble Sort: An Archaeological Algorithmic Analysis}


% author names and affiliations
% use a multiple column layout for up to three different
% affiliations
\author{\IEEEauthorblockN{Luis Alberto Guanuco}
\IEEEauthorblockA{Algortímos y Patrones de Software\\
Especialidad en Sistemas Embebidos\\
Instituto Universitario Aeronáutico}
}

% make the title area
\maketitle


\begin{abstract}
El presente documento realiza una revisión de la publicación
\emph{Bubble Sort: An Archaeological Algorithmic Analysis}. Se
rescatan los puntos más importantes de las investigaciones realizadas
por \emph{Owen Astrachan}. El eje central de la publicación es la
revisión histórica del ordenamiento burbuja (\emph{bubble
  sort}). Se buscan los orígenes del algoritmos y la injustificable
vigencia del mismo en los ámbitos académicos informáticos.
\end{abstract}

\IEEEpeerreviewmaketitle

\section{Introducción}
\label{sec:intro}

El autor lleva adelante una investigación profunda de los orígenes del
algoritmo \emph{Bubble Sort}. 

\section{Los orígenes del algoritmo}
\label{sec:origen-alg}

El primer registro del algoritmo se da en el año 1956. En aquella
oportunidad no se lo presenta como \emph{Bubble Sort}, se la expone
como \emph{sorting by exchange}. Las publicaciones que le siguieron a
esta primera aparición del algoritmo siguieron con refiriéndose como
sorting by exchange. 

Una primera aparición del nombre \emph{Bubble Sort} se da en el año
1962.  Kenneth Iverson es el matemático que lo denomina con este
nombre al algoritmo de ordenamiento burbuja.

En el año 1963 el algoritmo ingreso en los repositorios de la ACM
(Association for Computing Machinery) donde el nombre designado fue
\emph{Shuttle Sort}. Luego de su publicación se encontraron
definiciones anexas al algoritmo como \emph{not free from
  errors}. Esto es consecuencia de los errores encontrados en las
implementaciones sucesivas. 

\subsection{El código del algoritmo}
\label{sec:origen-code}

La descripción 
\begin{verbatim}
void BubbleSort(Vector a, int n)
{
  for(int j=n-1; j > 0; j--)
    for(int k=0; k < j; k++)
      if (a[k+1] < a[k])
        Swap(a,k,k+1);
}
\end{verbatim}

\section{Características de funcionamiento}
\label{sec:car-func}

\section{Conclusiones}
\label{sec:conc}


\begin{thebibliography}{1}
\bibitem{Astrachan}
  Owen~Astrachan, \emph{Bubble Sort: An Archaeological Algorithmic
    Analysis}. Computer Science Departament, Duke University. 
\end{thebibliography}

% that's all folks
\end{document}


